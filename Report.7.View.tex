\documentclass[13pt,a4paper]{report}
\usepackage{listings}

\begin{document}
\begin{titlepage}
	\centering
	\vspace{2cm}
	{\huge\bfseries Report labwork 7: write SQL commands using views \par}
	\vspace{2cm}
	{\Large\itshape Loc Thi Thuy Linh\par}
	\vfill
	supervised by\par
	Nghiem Thi Phuong \& Tran Giang Son \textsc{Brown}
	ICT Department, USTH\par
	\vfill
% Bottom of the page
	{\large \today\par}
\end{titlepage}

\section{ Who have the same name (last_name) as the managers of the “Finance” department? }
Analizing the query:
\\
\begin{itemize}
First, create view named 'manager_name' for list of the last name of managers of the “Finance” department. 
\begin{lstlisting}
	\item SQL Command
	CREATE VIEW manager_name AS
	SELECT last_name 
	FROM `employees` JOIN dept_manager ON employees.emp_no = dept_manager.emp_no
	JOIN departments ON departments.dept_no = dept_manager.dept_no
	WHERE departments.dept_name = 'Finance';
\end{lstlisting}
Then, create view named 'name_like_manager' for list of the name of employees who have the same name as the managers of the “Finance” department using query with the previous view.
\begin{lstlisting}

	\item SQL Command
	CREATE VIEW name_like_manager AS 
	SELECT CONCAT(first_name,' ',last_name) AS Fullname 
	FROM `employees` 
	WHERE employees.`last_name` IN (SELECT last_name FROM manager_name);
	
	SELECT * FROM name_like_manager;

	\item Output
	| Shirish Legleitner      |
	| Arvind Alpin            |
	| Shaw Alpin              |
	| Qiwen Alpin             |
	| Kristine Alpin          |
	| Conrado Alpin           |
	| Uri Legleitner          |
	| Danil Alpin             |
	| Constantijn Legleitner  |
	| Anoosh Legleitner       |
	| Aiichiro Legleitner     |
	| Gift Legleitner         |
	| Xinan Legleitner        |
	| Gilbert Alpin           |
	| Ramalingam Alpin        |
	| Ekawit Alpin            |
	| Hilary Legleitner       |
	| Geoffry Alpin           |
	| Heping Alpin            |
	| Hidde Alpin             |
	+-------------------------+
	408 rows in set (4.72 sec)
\end{lstlisting}
\end{itemize}

\section{ Who in the “Production” department were hired after the promotion of the last manager in that department? }
firstly, Create view named 'last_date_promotion' for command SELECT the date of the last manager who was promoted in "Production" department
\begin{itemize}
\begin{lstlisting}
	\item SQL Command
	CREATE VIEW last_date_promotion AS
	SELECT from_date FROM `dept_manager` ORDER BY from_date DESC LIMIT 0,1;
\end{lstlisting}
Finally, Create view name 'employee_list' for the selecting of employees who work for the “Production” department from the day in the previous view.
\end{itemize}
\begin{lstlisting}
	\item SQL Command
	CREATE VIEW employee_list AS
	SELECT CONCAT(first_name,' ',last_name) AS Fullname 
	FROM `employees` 
	JOIN `titles` ON employees.emp_no=titles.emp_no 
	WHERE titles.from_date >= (SELECT last_date_promotion.`from_date` FROM last_date_promotion);
\end{lstlisting}
Show table employee_list.
\begin{lstlisting}
	SELECT * FROM employee_list;
	\item Output
	| Masanao Ducloy                |
	| Uwe Uludag                    |
	| Katsuo Leuchs                 |
	| Shuichi Piazza                |
	| Martial Weisert               |
	| Gino Usery                    |
	| Yunming Mitina                |
	| Mohammed Pleszkun             |
	| Uri Juneja                    |
	| Gila Lukaszewicz              |
	| Rimli Dusink                  |
	| Bangqing Kleiser              |
	| Keiichiro Lindqvist           |
	| Khaled Kohling                |
	| Pohua Sichman                 |
	| DeForest Mullainathan         |
	| Dekang Lichtner               |
	| Zito Baaz                     |
	| Patricia Breugel              |
	| Sachin Tsukuda                |
	+-------------------------------+
	148376 rows in set (1.70 sec)

\end{lstlisting}

\section{ Find the average salary of each department, from highest to lowest. }
\begin{itemize}
To begin with, we create view named 'emp_avg_wage' for the selection of the average salary of each employees.
\begin{lstlisting}
	\item SQL Command
	CREATE VIEW emp_avg_wages AS
	SELECT employees.emp_no AS emp_no_new, AVG(salary) AS avg_emp 
	FROM `employees` 
	JOIN `salaries` ON employees.emp_no=salaries.emp_no 
	GROUP BY salaries.emp_no;
\end{lstlisting}
Then, we create view name 'dept_avg_wage' for querying the average salary of each department base on the average salary of each employee, from highest to lowest.
\begin{lstlisting}
	\item SQL Command
	CREATE VIEW dept_avg_wages AS
	SELECT departments.dept_name, AVG (emp_avg_wages.avg_emp) AS Average_Salary 
	FROM `emp_avg_wages`
	JOIN `dept_emp` ON emp_avg_wages.emp_no_new = dept_emp.emp_no 
	JOIN `departments` ON departments.dept_no = dept_emp.dept_no 
	GROUP BY departments.dept_no 
	ORDER BY Average_Salary DESC;
\end{lstlisting}
Show dept_avg_wages view.
\begin{lstlisting}
	SELECT * FROM dept_avg_wages;
	\item Output
	+--------------------+----------------+
	| dept_name          | Average_Salary |
	+--------------------+----------------+
	| Sales              | 78313.22247361 |
	| Marketing          | 69541.61771136 |
	| Finance            | 68061.43501801 |
	| Research           | 57322.03105659 |
	| Production         | 57253.31382027 |
	| Development        | 57152.20845497 |
	| Customer Service   | 56480.08591880 |
	| Quality Management | 54892.93507273 |
	| Human Resources    | 53214.29085744 |
	+--------------------+----------------+
	9 rows in set (18.77 sec)
	
\end{lstlisting}
\end{itemize}

\section{ Find the average salary of each type of Engineer, from highest to lowest. }
\begin{itemize}
First, create a view named 'emp_avg_wage_of_engineer_type' for the selection of the average salary of all employees belong to each type of Engineer
\begin{lstlisting}
	\item SQL Command
	CREATE VIEW emp_avg_wage_of_engineer_type AS
	SELECT AVG(salary) AS emp_avg_wage, titles.title AS Type_of_Engineer 
	FROM `employees` 
	JOIN `titles` ON employees.emp_no = titles.emp_no
	JOIN `salaries` ON employees.emp_no=salaries.emp_no 
	WHERE titles.title LIKE '%Engineer%' 
	GROUP BY employees.emp_no;
\end{lstlisting}
Finally, Creating a view name 'avg_wage_of_engineer_type' calculate the average salary of each type of Engineer, from highest to lowest.
\begin{lstlisting}
	\item SQL Command
	CREATE VIEW avg_wage_of_engineer_type AS
	SELECT emp_avg_wage_of_engineer_type.Type_of_Engineer, AVG(emp_avg_wage_of_engineer_type.emp_avg_wage) 
	AS average_type_engineer 
	FROM `emp_avg_wage_of_engineer_type`
	GROUP BY emp_avg_wage_of_engineer_type.Type_of_Engineer 
	ORDER BY average_type_engineer DESC;
\end{lstlisting}
Show dept_avg_wages view.
\begin{lstlisting}
	SELECT * FROM avg_wage_of_engineer_type;
	\item Output
	+--------------------+-----------------------+
	| Type_of_Engineer   | average_type_engineer |
	+--------------------+-----------------------+
	| Engineer           |        57007.79701364 |
	| Assistant Engineer |        56963.53043254 |
	| Senior Engineer    |        56936.31791254 |
	+--------------------+-----------------------+
	3 rows in set (41.41 sec)

\end{lstlisting}
\end{itemize}

\end{document}
