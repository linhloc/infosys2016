\documentclass{report}
\usepackage{listings}

\begin{document}

\begin{center}
Report labwork4: Write SQL queries Update And Delete
\end{center}
\section*{1. The problem}
Due to changes in company’s organization, the following departments are merged
\begin{itemize}
	\item Development (d005) and Research (d008) → Research and Development(d010)
	\item Marketing (d001) and Sales (d007) → Marketing and Sales (d011)
	\item neccessary SQL queries to implement these changes to the employees database
\end{itemize}
Requirements:
+ Describe your main steps to reach the above goal
+ Explain why you need to perform these steps
+ Write your SQL commands
+ Capture output of each SQL command


\section*{2. Solving the problem}
\subsection*{2.1. Main steps to reach the goal}

\indent First, creating 2 new departments on the table 'departments' which are Research and Development(d010) department and Marketing and Sales (d011) department respectively use the INSERT command.
\indent Next, updating the old departments into the new departments on the table 'dept_emp' and the table 'dept_manager' uses 'dept_no' property.
\indent Finally, deleting the tuples of the four departments on the table which are Development (d005), Research (d008), Marketing (d001), and Sales (d007).

\subsection*{2.2. Explaining}
\indent We need to create the new departments first before merging the old departments into a new one which is the update step. The final step which deletes the tuples is option.

\subsection*{2.3. The SQL commands and the output of each SQL command}
\begin{lstlisting}
SQL INSERT command: 
INSERT INTO `departments` (dept_no, dept_name) VALUES ('d010','Research and Development'), ('d011','Marketing and Sales');
+Output:
Query OK, 2 rows affected (0.01 sec)
Records: 2  Duplicates: 0  Warnings: 0

SQL UPDATE command: 
UPDATE IGNORE `dept_emp` SET dept_emp.dept_no="d010" WHERE dept_emp.dept_no = "d005";
+ Output:
Query OK, 85707 rows affected (1.73 sec)
Rows matched: 85707  Changed: 85707  Warnings: 0

SQL UPDATE command:
UPDATE IGNORE `dept_emp` SET dept_emp.dept_no="d010" WHERE dept_emp.dept_no = "d008";
+ Output:
Query OK, 15892 rows affected (0.42 sec)
Rows matched: 21126  Changed: 15892  Warnings: 0

// When I used this following command, I didn't get the same total number of rows affected in comparing with the 2 previous SQL commands (101599 rows affected).
// UPDATE IGNORE `dept_emp` SET dept_emp.dept_no="d010" WHERE dept_emp.dept_no IN ('d005','d008');
// Output:
// Query OK, 101581 rows affected (2.83 sec)
// Rows matched: 106815  Changed: 101581  Warnings: 0

UPDATE command:
UPDATE IGNORE `dept_emp` SET dept_emp.dept_no="d011" WHERE dept_emp.dept_no = "d001";
+ Output:
Query OK, 20211 rows affected (0.48 sec)
Rows matched: 20211  Changed: 20211  Warnings: 0

UPDATE command:
UPDATE IGNORE `dept_emp` SET dept_emp.dept_no="d011" WHERE dept_emp.dept_no = "d007";
+ Output:
Query OK, 48593 rows affected (1.30 sec)
Rows matched: 52245  Changed: 48593  Warnings: 0

UPDATE command on table "dept_manager":

UPDATE IGNORE `dept_manager` SET dept_manager.dept_no="d010" WHERE dept_manager.dept_no = "d005";
+ Output
Query OK, 2 rows affected (0.02 sec)
Rows matched: 2  Changed: 2  Warnings: 0

UPDATE IGNORE `dept_manager` SET dept_manager.dept_no="d010" WHERE dept_manager.dept_no = "d008";
+ Output
Query OK, 2 rows affected (0.02 sec)
Rows matched: 2  Changed: 2  Warnings: 0

UPDATE IGNORE `dept_manager` SET dept_manager.dept_no="d011" WHERE dept_manager.dept_no = "d001";
+ Output
Query OK, 2 rows affected (0.00 sec)
Rows matched: 2  Changed: 2  Warnings: 0

UPDATE IGNORE `dept_manager` SET dept_manager.dept_no="d011" WHERE dept_manager.dept_no = "d007";
+ Output:
Query OK, 2 rows affected (0.00 sec)
Rows matched: 2  Changed: 2  Warnings: 0

DELETE command:
DELETE FROM `departments` WHERE departments.dept_no="d001" OR departments.dept_no="d005" OR departments.dept_no="d007" OR departments.dept_no="d008";
+ Output:
Query OK, 4 rows affected (0.02 sec)

\end{lstlisting}

\end{document}
